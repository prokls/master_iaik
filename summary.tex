\renewcommand*\chappic{}
\renewcommand*\chapquote{}
\chapter{Summary and Future Work}
\label{ch:summary}
%
\section{Related work}
\label{sec:results-related}
%
Already in my bachelor thesis~\cite{bach} we tried to integrate a SAT solver into
a dedicated, automated tool finding collisions in hash functions using differential
cryptanalysis. This approach was not very successful as restarts between hash
algorithm rounds implied that intermediate results by the SAT solver got lost.
This was one motivation for this master thesis: We represent all details in CNF.

Research was already done by Ilya Mironov and Lintao Zhang~\cite{mironov2006applications}
to apply SAT solvers to differential cryptanalysis specifically to find hash collisions
in MD4. Their approach corresponds to our basic approach applied to Testcase~A
(compare with Section~\ref{sec:tcA}), where all differences are assigned.
Therefore our results in Table~\ref{tab:tcA-results} within one minute are
comparable with their evaluations within ten minutes.
We can clearly see how SAT technology and CPU performance
has progressed over 10 years since publication of this paper.

Finding hash collisions in Testcases~B, C, 21, 23 and 24 seems
to be a novel result of this master thesis and has not been found
in related work.

\section{Conclusion}
\label{sec:conclusion}
%
We successfully found full-round hash collisions for MD4
using SAT solvers mentioned in Section~\ref{sec:sat-solvers}.
We modified the lingeling SAT solver to improve our
runtime results further and found 24-round hash collisions
for SHA-256. Our attack starting points for MD4 ---
Testcases~\ref{fig:tcA}, \ref{fig:tcB} and \ref{fig:tcC} ---
are based on the work by Yusuke Naito, Yu Sasaki, Noboru Kunihiro and
Kazuo Ohta~\cite{sasaki2007new}. Our starting points for SHA-256
--- Testcases~\ref{fig:tc18}, \ref{fig:tc21}, \ref{fig:tc23}
and \ref{fig:tc24} --- are based on the work by Ivica Nikoli{\'c}
and Alex Biryukov~\cite{nikolic2008collisions}.

\section{Contributions}
\label{sec:contributions}
%
To encourage future work, the source code and data resulting from this
thesis is available online.
It allows the reader to run the experiments again and verify our claims.
We did our best to describe our hardware setup as accurately as possible.
At the following website, any results part of this project are collected:

\begin{quote}
  \url{http://lukas-prokop.at/proj/megosat/}
\end{quote}

% \noindent
% Several subprojects are part of this master thesis:
% \begin{description}
% \item[algotocnf]\hfill{} \\
%   A python library implementing the encoding described in Chapter~\ref{ch:enc}. \\[4pt]
%   \textbf{Python3 library and program:} \url{https://github.com/prokls/algotocnf}
% \item[cnf-hash]\hfill{} \\
%   A standardized way to produce a hash value for CNF files \\[4pt]
%   \textbf{Go implementation:} \url{https://github.com/prokls/cnf-hash-go} \\
%   \textbf{Python3 implementation:} \url{https://github.com/prokls/cnf-hash-py} \\
%   \textbf{Testsuite:} \url{https://github.com/prokls/cnf-hash-tests2}
% \item[cnf-analysis]\hfill{} \\
%   Evaluate SAT features for a given CNF file. \\[4pt]
%   \textbf{Go implementation:} \url{https://github.com/prokls/cnf-analysis-go} \\
%   \textbf{Python3 implementation:} \url{https://github.com/prokls/cnf-analysis-py} \\
%   \textbf{Testsuite:} \url{https://github.com/prokls/cnf-analysis-tests}
% \end{description}

\section{Future work}
\label{sec:future}
%
Future work might want to consider our design decisions made in
Chapter~\ref{ch:enc}.

In general, it would be interesting to generate our testcase for a broader
set of differential characteristics. Probably we can come up with empirical
results showing a relation between the size of unspecified areas in the
differential characteristic and the evaluated runtimes.

Furthermore, many SAT-related effort could be put to thoroughly discuss
why differential description provides such a significant performance improvement.
This necessarily means the SAT solver is generally not capable of deriving
the useful clauses, differential description provides. The resulting
numbers of restarts could also be subject of further research.

Lingeling ats1o1 was experimentally modified to define a separate set of
variables to be evaluated first. This approach seemed promising and
should be subject to future SAT research.

As far as cnf-analysis is concerned, the project aims to extend to
a larger set of SAT features and feedback by SAT solver developers
is appreciated. Our main contribution is a search
interface to search for SAT features given the cnfhash of a CNF file.
We hope to get in touch with new SAT feature ideas and SAT benchmark files.
