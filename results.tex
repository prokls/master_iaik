\renewcommand*\chappic{img/megosat.pdf}
\renewcommand*\chapquote{}
\renewcommand*\chapquotesrc{}
\chapter{Results}
\label{ch:results}
%
In Section~\ref{ch:features} we looked at SAT features which to some extent characterize
a SAT problem. In Section~\ref{ch:enc} we discussed problem encoding for our hash collision
attacks. In this section, we want to look at the feature value results and runtime results
we retrieved.

\section{Evaluating SAT features}
\label{sec:results-features}

In the introduction of chapter~\ref{ch:features} we posed 8 questions.
In the following we want to answer them with the data provided by the
cnf-analysis project.

\begin{description}
\item[Given an arbitrary literal. What is the percentage it is positive?]
  We look at every clause and determine the ratio of positive to the total number of literals.
  We determine the mean per CNF file and the mean among all CNF files
  and retrieve a value of $0.48$ meaning that 48~\% of the literals are positive.
\item[What is the variables / clauses ratio?]
  In average a CNF file has 12,219 variables and 89,541 clauses.
  Its variables-clauses ratio is 0.1365.
\item[How many literals occur only once either positive or negative?]
  In average there are 36 existential literals per CNF file,
  but its standard deviation of 967 is very high.
\item[What is the average and longest clause length among CNF benchmarks?]
  The average clause length is 3.04 with a standard deviation of 0.99
  and the longest clause length found was 61,473. Long clauses are typically
  outliers excluding specific assignments.
\item[How many Horn clauses exist in a CNF?]
  In average 29,994 goal clauses and 31,315 definite clauses exist
  with an average number of 83649 clauses in a CNF file.
\item[Are there any tautological clauses?]
  In a file, 1679 tautological literals have been found. However,
  its mean is 0.07 with a standard deviation of $9.63$ meaning that tautological
  clauses are very rare.
\item[Are there any CNF files with more than one connected variable component?]
  Indeed, an average CNF file contains 67.07 connected variable components.
  However, its median is 1 anyway implying that at least half of the CNF files
  have only 1 connected variable component.
\item[How many variables of a CNF are covered by unit clauses?]
  In average 124 variables are covered by unit clauses. This is an insignificant
  number compared to 12,219 variables in an average CNF.
\end{description}

Do cryptoproblems distinguish from average problems?

TODO

Do our generated problems distinguish from these average problems?

\begin{itemize}
  \item In average our (MD4 and SHA-256 hash collision) problems contain
        84,821 variables and 515,646 clauses meaning that our problems
        are comparably large. It is important to point out that the
        problem size does not necessarily correlate with the difficulty
        of the problem in SAT solving.
  \item TODO
% TODO
%  \item Due to the cryptographic high diffusion we expected a 
\end{itemize}

TODO: illustrate in our benchmarks, how do the CNF features behave when transitioning from original to simplificiation or diff-desc

\section{Finding hash collisions}
\label{sec:results-attacks}

TODO: is simplification worth it?

Appendix~\ref{app:runtimes} provide a more exhaustive list of runtimes retrieved.

\subsection{Attacking MD4}
\label{sec:results-md4}
%
In Section~\ref{sec:enc-algotocnf} we introduced a basic encoding involving two hash algorithm
instances and difference variables. Constraints resulting from the hash algorithm description
and given differential characteristic are added.

We considered MD4 testcases~A, B and C (compare with Appendices~\ref{tab:tcA}, \ref{tab:tcB} and \ref{tab:tcC})
and generated the corresponding CNF files. The SAT solvers mentioned in Section~\ref{sec:sat-solvers}
were used to evaluate whether the problem is solvable in reasonably time. For every testcase we
defined a time limit of at most 1 day (i.e. 86,400 seconds). A timeout is denoted by \timeout.
Some testcases listed have been evaluated for a larger time limit.

\begin{table}[!h]
  \begin{center}
    \begin{tabular}{lcc}
      SAT solver                & testcase      & runtime (in seconds) \\
    \hline
      minisat~2.2.0             & MD4, A        & 65 \\
      cryptominisat~4.5.3       & MD4, A        & 24 \\
      cryptominisat~5.0.0       & MD4, A        & 29 \\
      glucose~4.0               & MD4, A        & 10 \\
      glucose-syrup~4.0         & MD4, A        & 31 \\
      lingeling-ats1            & MD4, A        & TODO \\
      lingeling-ats1o1          & MD4, A        & 18 \\
      lingeling-ats1o2          & MD4, A        & TODO \\
      lingeling-ats1o4          & MD4, A        & 125,745 \\
      plingeling-ats1o1         & MD4, A        & 88 \\
      treeneling-ats1o1         & MD4, A        & 64
    \end{tabular}
    \caption{Runtimes for MD4 testcase A with various SAT solvers}
    \label{tab:md4-A-runtimes}
  \end{center}
  \begin{center}
    \begin{tabular}{lcc}
      SAT solver                & testcase      & runtime (in seconds) \\
    \hline
      minisat~2.2.0             & MD4, B        & 7,817 \\
      cryptominisat~4.5.3       & MD4, B        & \timeout \\
      cryptominisat~5.0.0       & MD4, B        & 571 \\
      glucose~4.0               & MD4, B        & \timeout \\
      glucose-syrup~4.0         & MD4, B        & \timeout \\
      lingeling-ats1            & MD4, B        & TODO \\
      lingeling-ats1o1          & MD4, B        & 257 \\
      lingeling-ats1o2          & MD4, B        & TODO \\
      lingeling-ats1o4          & MD4, B        & TODO \\
      plingeling-ats1o1         & MD4, B        & 1,860 \\
      treeneling-ats1o1         & MD4, B        & 12,574
    \end{tabular}
    \caption{Runtimes for MD4 testcase B with various SAT solvers}
    \label{tab:md4-B-runtimes}
  \end{center}
  \begin{center}
    \begin{tabular}{lcc}
      SAT solver                & testcase      & runtime (in seconds) \\
    \hline
      minisat~2.2.0             & MD4, C        & 19,683 \\
      cryptominisat~4.5.3       & MD4, C        & \timeout \\
      cryptominisat~5.0.0       & MD4, C        & 1064 \\
      glucose~4.0               & MD4, C        & \timeout \\
      glucose-syrup~4.0         & MD4, C        & \timeout \\
      lingeling-ats1            & MD4, C        & TODO \\
      lingeling-ats1o1          & MD4, C        & TODO \\
      lingeling-ats1o2          & MD4, C        & TODO \\
      lingeling-ats1o4          & MD4, C        & TODO \\
      plingeling-ats1o1         & MD4, C        & TODO \\
      treeneling-ats1o1         & MD4, C        & TODO
    \end{tabular}
    \caption{Runtimes for MD4 testcase C with various SAT solvers}
    \label{tab:md4-C-runtimes}
  \end{center}
\end{table}

In Table~\ref{tab:md4-A-runtimes} it can be seen that the problem can be tackle
by all SAT solvers. lingeling-ats1o4 being a outlier can be ignored, because this
release is majorily concerned with providing debug information. As such it only
shows that printing information to stdout can make a major performance difference.

In Table~\ref{tab:md4-B-runtimes} we see that CryptoMiniSat~4.5.3, glucose~4.0
and glucose-syrup~4.0 couldn't solve the problem within the time limit of 1~day.
However, other SAT solver were still able to find a hash collision. Please recognize
that Table~\ref{tab:md4-A-runtimes} provides runtime results for the testcase
given in Table~\ref{tab:tcA}; equivalently Table~\ref{tab:md4-B-runtimes} for
Table~\ref{tab:tcB} and Table~\ref{tab:md4-C-runtimes} for Table~\ref{tab:tcC}.
Its caption gives an intuition how testcase B is more difficult than A and
C is more difficult than B.

We end up with the result, that the hash collision given in Table~\ref{tab:tcC}
can be solved by a limited set of modern SAT solvers. Of course the cryptanalyst
needs to figure out good starting points for the hash collision and encode them
in the differential characteristic, but this task is still considered practical,
because this task can be easily automated.

\subsection{Improvements with differential description}
\label{sec:result-diff-desc}
%
Our next goal was to scale up to a more difficult problem. We considered SHA-256
which has a much larger internal state (at least by factor 2). So finding a hash
collision is more difficult and we considered further strategies.

Consider testcases 18~\ref{tab:18-t9}, 21~\ref{tab:21-t9}, 23~\ref{tab:23-t9}
and 24~\ref{tab:24-t9}. The number indicates how many steps are covered by this
testcase. We also tested a 27-round variant, but it is not listed here.
Most SAT solvers could not solve this problem in feasible time. Therefore we
excluded it from our list.

\begin{table}[!h]
  \begin{center}
    \begin{tabular}{lcc}
      SAT solver                & testcase      & runtime (in seconds) \\
    \hline
      minisat~2.2.0             & SHA-256, 18   & TODO \\
      cryptominisat~4.5.3       & SHA-256, 18   & TODO \\
      cryptominisat~5.0.0       & SHA-256, 18   & TODO \\
      glucose~4.0               & SHA-256, 18   & TODO \\
      glucose-syrup~4.0         & SHA-256, 18   & TODO \\
      lingeling-ats1            & SHA-256, 18   & TODO \\
      lingeling-ats1o1          & SHA-256, 18   & 25 \\
      lingeling-ats1o2          & SHA-256, 18   & TODO \\
      lingeling-ats1o4          & SHA-256, 18   & TODO \\
      plingeling-ats1o1         & SHA-256, 18   & TODO \\
      treeneling-ats1o1         & SHA-256, 18   & TODO
    \end{tabular}
    \caption{Runtimes for SHA-256 testcase 18 with various SAT solvers}
    \label{tab:SHA-256-18-runtimes}
  \end{center}
  \begin{center}
    \begin{tabular}{lcc}
      SAT solver                & testcase      & runtime (in seconds) \\
    \hline
      minisat~2.2.0             & SHA-256, 21   & TODO \\
      cryptominisat~4.5.3       & SHA-256, 21   & TODO \\
      cryptominisat~5.0.0       & SHA-256, 21   & TODO \\
      glucose~4.0               & SHA-256, 21   & TODO \\
      glucose-syrup~4.0         & SHA-256, 21   & TODO \\
      lingeling-ats1            & SHA-256, 21   & TODO \\
      lingeling-ats1o1          & SHA-256, 21   & 27,511 \\
      lingeling-ats1o2          & SHA-256, 21   & TODO \\
      lingeling-ats1o4          & SHA-256, 21   & TODO \\
      plingeling-ats1o1         & SHA-256, 21   & TODO \\
      treeneling-ats1o1         & SHA-256, 21   & TODO
    \end{tabular}
    \caption{Runtimes for SHA-256 testcase 21 with various SAT solvers}
    \label{tab:SHA-256-21-runtimes}
  \end{center}
  \begin{center}
    \begin{tabular}{lcc}
      SAT solver                & testcase      & runtime (in seconds) \\
    \hline
      minisat~2.2.0             & SHA-256, 23   & TODO \\
      cryptominisat~4.5.3       & SHA-256, 23   & TODO \\
      cryptominisat~5.0.0       & SHA-256, 23   & TODO \\
      glucose~4.0               & SHA-256, 23   & TODO \\
      glucose-syrup~4.0         & SHA-256, 23   & TODO \\
      lingeling-ats1            & SHA-256, 23   & TODO \\
      lingeling-ats1o1          & SHA-256, 23   & 59,227 \\
      lingeling-ats1o2          & SHA-256, 23   & TODO \\
      lingeling-ats1o4          & SHA-256, 23   & TODO \\
      plingeling-ats1o1         & SHA-256, 23   & TODO \\
      treeneling-ats1o1         & SHA-256, 23   & TODO
    \end{tabular}
    \caption{Runtimes for SHA-256 testcase 23 with various SAT solvers}
    \label{tab:SHA-256-23-runtimes}
  \end{center}
  \begin{center}
    \begin{tabular}{lcc}
      SAT solver                & testcase      & runtime (in seconds) \\
    \hline
      minisat~2.2.0             & SHA-256, 24   & TODO \\
      cryptominisat~4.5.3       & SHA-256, 24   & TODO \\
      cryptominisat~5.0.0       & SHA-256, 24   & TODO \\
      glucose~4.0               & SHA-256, 24   & TODO \\
      glucose-syrup~4.0         & SHA-256, 24   & TODO \\
      lingeling-ats1            & SHA-256, 24   & TODO \\
      lingeling-ats1o1          & SHA-256, 24   & 65,956 \\
      lingeling-ats1o2          & SHA-256, 24   & TODO \\
      lingeling-ats1o4          & SHA-256, 24   & TODO \\
      plingeling-ats1o1         & SHA-256, 24   & TODO \\
      treeneling-ats1o1         & SHA-256, 24   & TODO
    \end{tabular}
    \caption{Runtimes for SHA-256 testcase 24 with various SAT solvers}
    \label{tab:SHA-256-24-runtimes}
  \end{center}
\end{table}

In Tables~\ref{tab:SHA-256-18-runtimes} to \ref{tab:SHA-256-24-runtimes}
we see several results which illustrate TODO


Followingly we added the clauses which directly encode how differences
propagate in the hash algorithm; namely \enquote{differential description}
of Section~\ref{sec:enc-diff-desc}. If we compare the data, we can see
TODO

We continued by trying the influence the guessing strategy.

\subsection{Modifying the guessing strategy}
\label{sec:results-guessing}
%
As pointed out in Section~\ref{sec:enc-diff-desc-ocnf}, a best practice law of
differential cryptanalysis states that difference variables should be
assigned first. Afterwards propagation of actual values for the two
instances can take place.

To enforce such a strategy, we tried several approaches:
\begin{enumerate}
  \item
    Armin Biere provided us with a custom release of lingeling
    which enforces that a special set of variables is evaluated
    first. It is important that the CNF is still solved
    with usual SAT solver heuristics, because enforcing
    assignment of one variable after another leads to an increase
    in backtracking steps and restarts. Hence, we consider this
    release as a nice tradeoff. Difference variables are assigned
    \enquote{as early as possible, as late as necessary}.
  \item
    Given this custom SAT solver we considered a SAT design which
    requires the SAT solver to prefer a certain. This particular
    design with a special Boolean variable is explained in
    Section~\ref{sec:enc-diff-desc-eo}.
\end{enumerate}

\section{Related work}
\label{sec:results-related}
%
In my bachelor thesis~\cite{bach} we tried to integrate a SAT solver into
our department's existing tool which encodes propagation explicitly. This approach
was not very successful as restarts between hash algorithm rounds implied that
intermediate results by the SAT solver get lost.

Research was already done by Ilya Mironov and Lintao Zhang~\cite{mironov2006applications}
to apply SAT solvers to differential cryptanalysis specifically to find hash collision
in MD4. However whereas their basic approach seems to correspond to our approach described
in Section~\ref{sec:enc-original}, our implementation uses additional SAT design tweaks
to improve our results. Also because 10 years have gone since publication, SAT technology
has progressed and modern SAT solvers on modern hardware provide better results.

\section{Conclusion}
\label{sec:conclusion}

We successfully found hash collisions for round-reduced MD4 and SHA-256.

\section{Contributions}
\label{sec:contributions}
%
To strengthen Reproducible Research, the source code and data resulting from this thesis is available online.
It allows the reader to run the experiments again and verify our claims.
We did our best to describe our hardware setup as accurately as possible.
At the following website, any results part of this project are collected:

\begin{quote}
  \url{http://lukas-prokop.at/proj/megosat/}
\end{quote}

Several subprojects are part of this master thesis:
\begin{description}
\item[algotocnf]\hfill{} \\
  A python library implementing the encoding described in chapter~\ref{ch:enc}. \\[4pt]
  \textbf{Python3 library and program:} \url{https://github.com/prokls/algotocnf}
\item[cnf-hash]\hfill{} \\
  A standardized way to produce a unique hash for CNF files \\[4pt]
  \textbf{Go implementation:} \url{https://github.com/prokls/cnf-hash-go} \\
  \textbf{Python3 implementation:} \url{https://github.com/prokls/cnf-hash-py} \\
  \textbf{Testsuite:} \url{https://github.com/prokls/cnf-hash-tests2}
\item[cnf-analysis]\hfill{} \\
  Evaluate SAT features for a given CNF file. \\[4pt]
  \textbf{Go implementation:} \url{https://github.com/prokls/cnf-analysis-go} \\
  \textbf{Python3 implementation:} \url{https://github.com/prokls/cnf-analysis-py} \\
  \textbf{Testsuite:} \url{https://github.com/prokls/cnf-analysis-tests}
\end{description}
