\renewcommand*\chappic{img/encoding.pdf}
\renewcommand*\chapquote{There is concensus that encoding techniques usually have a dramatic impact on the efficiency of the SAT solver}
\renewcommand*\chapquotesrc{Magnus Bj\"ork}
\chapter{Problem encoding}
\label{ch:enc}

We already discussed how SAT solvers work and which input they take.
We also sketched how hash algorithm properties got broken using
differential cryptanalysis. In this section we combine those
subjects and describe how we designed an attack setting.

Every section represents a major approach whereas subsections
represent derivatives of this approach with minor tweaks.

\section{STP approach}
\label{sec:enc-stp}
%
Our first approach started with STP~\cite{stp} initially written by Vijay Ganesh and David L. Dill.
It is currently maintained by Mate Soos.

First we wrote an implementation using the CVC language to model the MD4 hash algorithm.
Reimplementing hash algorithms in CVC language (i.e. generating the corresponding code)
seemed cumbersome and we switched to the Python binding. With little modifications to
a working pure-Python MD4 implementation, the prototype was working.

However, STP was not fruitful for us, because we needed good control over the SAT encoding
which we expected to have a major influence on the performance. We used minisat as SAT solver
in the backend, but STP allows to exchange it for CryptoMiniSat which is a more modern and
versatile SAT solver. In the end this straight-forward approach worked nicely for MD4,
but we needed a different toolchain for SHA-256.

\section{algotocnf}
\label{sec:enc-algotocnf}
%
We implemented our own library \emph{algotocnf} to achieve greater flexibility in SAT design.

\subsection{Two instances and its difference}
\label{sec:enc-original}
%
\index{algotocnf}
algotocnf generated a CNF for a given hash algorithm implementation which is fed with
a symbolic variable. Only integer operations have been implemented.

Our tool \emph{algotocnf} implements the following strategy:
\begin{enumerate}
  \item Take a differential characteristic and the hash algorithm as input.
  \item For two instances of the hash algorithm, initialize symbolic variables
    (implemented as bitvectors) representing the bytes used to compute
    the hash value.

    Every bit is therefore represented as a boolean variable.
    If you apply addition, operator overloading in python will ensure
    that clauses are generated to describe the addition consisting of
    \boolf{XOR}s and \boolf{MAJ}s. Every operation is modelled as assignment.
    Hence an operation operating on a few Boolean variables is equivalent
    to a single variable which represents the result.
    Equivalently other operations related to integers are implemented as well.
  \item Constants used in the implementation are automatically converted
    to bitvectors with unit clauses.
  \item After running the hash algorithm with bitvectors per instance,
    all constraints related to the hash algorithm are added.
  \item Followingly the differential characteristic is read. Values such as $A_i$
    represent intermediate states of bitvectors. Therefore the corresponding bitvectors
    are looked up and equivalences with temporary bitvectors are added.
    Those temporary bitvectors are initialized with all constraints resulting from
    the bit conditions of this bitvector.
    In conclusion all constraints resulting from the differential characteristic are added.
  \item Finally the SAT solver is called. The CNF was mostly solved on a cluster
    specified in Appendix~\ref{app:setup}.
  \item Afterwards the program is run again to
    create the exactly same problem instance and the solver's solution replaces
    symbolic values with actual Boolean values. The resulting differential
    characteristic is parsed backed and printed out as differential characteristic
    where expectedly many bit conditions have been strengthened.
\end{enumerate}

We think \emph{algotocnf} mainly differs from other SAT abstraction tools because
of its implementation of differential logic. When adding clauses resulting from
the differential characteristic as constraints, the question arises how those
bit conditions are encoded. Essentially, we have only boolean values available,
but bit conditions tell constraints such as \enquote{a difference is given,
but the actual value is unknown}.

It seemed trivial to add a \emph{difference variable} for every pair of boolean
values representing a bit in the two instances. Furthermore the difference
variable $\Delta x$ is connected by a XOR with the variables of the pair $(x, x')$.
\[ \Delta x = x \oplus x' \]
Therefore it should be trivial for a preprocessor to simplify the formula
appropriately or actually we don't expect runtime differences for the larger
amount of variables.

And finally we expect the CNF to inherit a property of hash functions.
Inputs are provided into the hash algorithm and strongly intermingled
with other values. This should result in a high diffusion and almost every
variable is expected to share a clause with another variable.

The design decisions of this encoding are fundamental to the
resulting runtime discussed in chapter~\ref{ch:results}.

\subsection{Adding the differential description}
\label{sec:enc-diff-desc}
%
Using the approach in the previous section, we were able to find actual MD4 collisions
using a SAT solver. A SHA256 implementation followed which obviously lead to worse
runtime results, because the internal state of SHA-256 is much larger (by a factor
of at least 2). Can we further improve the runtime of the SAT solver?

Because we work with bitvectors and apply high-level operations like \boolf{MAJ} or addition,
we can additionally implement how differences in those operations propagate.
Magnus Daum's thesis on \enquote{Cryptanalysis of Hash Functions of the
MD4-Family}~\cite[Table 4.4]{daum2005cryptanalysis} discusses how differences propagate in \boolf{MAJ} and
\boolf{ITE} functions. Trivially, \boolf{XOR}s propagate differences the way they are~\footnote{
A difference in the arguments of two \boolf{XOR} instances remains the same difference
after applying \boolf{XOR} to each instance}.

This approach explicitly models differentiable behavior, which should be deducible
by the SAT solver itself. However, this lead to a major speedup which can be observed
in the runtime results of chapter~\ref{ch:results}.

\newpage
\subsection{Evaluating difference variables first}
\label{sec:enc-diff-desc-ocnf}
%
\begin{prop}
  Deriving difference values first, followed by actual bit values for the two instances,
  leads to a speedup.
\end{prop}

This proposed principle is fundamental to differential cryptanalysis. A previous tool
at our department implements propagation of hash algorithm values without a SAT solver
and this strategy is essential to good performance. This strategy was introduced
in the very early days of differential cryptanalysis and was also used by Wang et
al.~\cite{wang2004} to find their hash collisions.

So basically in terms of SAT solvers we want to guess values for differential variables
first and once all have been assigned, we try to find actual values for the two
hash algorithm instances.

It is important to point out that DIMACS does not specify a way to annotate Boolean
variables. As such that SAT solver cannot distinguish between difference variables
and variables of the instances. Therefore implementing this approach requires a custom
SAT solver which is given with lingeling~ats1o1.

Another proposition is important for this approach:

\begin{prop}
  Guessing difference values false first, followed by true,
  should solve hash collision problems faster.
\end{prop}

This proposition is justified by the desire to find as little differences as possible
in a hash collision to increase the probability of values cancelling each other out
in the later rounds.

\subsection{A light-weight approach}
\label{sec:enc-ocnf}
%
In this tweak we made a step back and considered the ideas of the previous
section but neglected the differential description. This approach was interesting
to see the effect of differential descriptions.

\subsection{Influencing the evaluation order}
\label{sec:enc-diff-desc-eo}
%
We took to idea to influence the evaluation order to the next level,
by applying the following design:

Let $\Delta x$ be the difference variable of pair $(x, x')$. We introduce a new boolean
variable $x^*$. We add clause
\[ x^* = (\Delta x \land x) \]
and explicitly tell the SAT solver to guess on $x^*$ before guessing on $\Delta x$, $x$ or $x'$.

The SAT solver will assign $x'=0$ first, because of the evaluation order. So either $\Delta x$
or $x$ must be false. $\Delta x$ is assigned false, because as difference variable it has a higher
priority over $x$. Equivalently for $x'=1$ we have $\Delta x$ needs to be true. So we actually
achieve an early guess on the difference variable.

This another approach evaluates in the results chapter.
