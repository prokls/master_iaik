\renewcommand*\chappic{img/intro.pdf}
\renewcommand*\chapquote{}
\renewcommand*\chapquotesrc{}
\chapter{Introduction}
\label{ch:intro}
%
Hash functions are used as cryptographic primitives in many applications and protocols.
They take an arbitrary input message and provide a hash value. Input message and hash value
are considered as byte strings in a particular encoding.
The hash value is of fixed length and satisfies several properties which make it useful
in a variety of applications.

In this thesis we will consider the hash algorithms MD4 and SHA-256 and represent
differential characteristics of hash collisions as SAT problem. If and only if
satisfiability is given, the particular differential state is achievable
using two different inputs leading to the same output. As far as SAT solvers
return an actual model satisfying that state, we get a complete hash collision
which can be verified and visualized.
If the internal state of the hash algorithm is too large, the attack can be easily
computationally simplified by modelling only a subset of steps of the hash algorithm
or changing the modelled differential path.

Gaining experience with these kind of problems with previous non-SAT-based tools
we try to apply best practices to a satisfiability setting.
We will discuss which SAT techniques lead to best performance characteristics
for our MD4 and SHA-256 testcases.

\newpage  % FIX manual pagebreak
\section{Preliminaries}
\label{sec:intro-prelim}
%
\index{Hash value}
\index{Preimage}
\index{Hash function}
\begin{defi}[Hash function]
  A \emph{hash function} is a mapping $h: X \to Y$ with $X = \left\{0,1\right\}^*$ and
  $Y = \left\{0,1\right\}^n$ for some fixed $n \in {\mathbb{N}}_{\geq 1}$.
  \begin{itemize}\itemsep0pt
    \item Let $x \in X$, then $h(x)$ is called \emph{hash value of $x$}.
    \item Let $h(x) = y \in Y$, then $x$ is called \emph{preimage of $y$}.
  \end{itemize}
\end{defi}

One example showing the use of hash functions as primitives is  % TODO
 PKCS~\#5
specified in RFC~2898~\cite{rfc2898}. Section~5.2 specifies PBKDF1 and PBKDF2
using an arbitrary pseudorandom function to derive password-based keys.
Hash algorithms can be used as those pseudorandom functions.
Given a minimum iteration count of $1000$, as defined in Section~4.2,
yields the additional requirement that fast computation of hash values
for given $x \in X$ is desirable.

A hash function has to satisfy the following security requirements:

\index{Preimage resistance}
\begin{defi}[Preimage resistance]
  Given $y \in Y$,
  a hash function $h$ is \emph{preimage resistant} iff it is computationally infeasible
  to find $x \in X$ such that $h(x) = y$.
\end{defi}

\index{Second-preimage resistance}
\begin{defi}[Second-preimage resistance]
  Given $x \in X$,
  a hash function $h$ is \emph{second-preimage resistant} iff it is computationally infeasible
  to find $x_2 \in X$ with $x \neq x_2$ such that $h(x) = h(x_2)$.
  $x_2$ is called \emph{second preimage}.
\end{defi}

\index{Collision resistance}
\begin{defi}[Collision resistance]
  A hash function $h$ is \emph{collision resistant} iff it is computationally infeasible to
  find any two $x \in X$ and $x_2 \in X$ with $x \neq x_2$ such that $h(x) = h(x_2)$.
\end{defi}

As far as hash functions accept input strings of arbitrary length, but return a fixed
size output string, existence of collisions is unavoidable~\cite{schlaffer}.
However, good hash functions make it very difficult to determine collisions or preimages.

Most hash functions apply padding to their input to normalize the input size to a multiple
of its block size before running a round function. In the following we always consider
input of block size meaning the round function is run only for one block and padding is
always considered part of this input message. Padding is negligible, because given two
colliding blocks, we can add another incomplete block. Padding will occur only in the
second block and with the same values in the second blocks it yield the same padded value.
This results in a length extension attack, making input padding negligible for cryptanalysis.
% TODO: only applies to single-pipe constructions like MD4

\section{Cryptanalysis of Hash Functions}
\label{sec:intro-cryptanalysis}
%
In August 2004, Wang et al. published results at Crypto'04~\cite{wang2004} which revealed
that MD4, MD5, HAVAL-128 and RIPEMD can be broken practically using differential cryptanalysis.
Their work is based on preliminary work by Hans Dobbertin~\cite{Dobbertin1998}.
On an IBM~P690 machine, an MD5 collision can be computed in about one hour using this approach.
Collisions for HAVAL-128, MD4 and RIPEMD were found as well. Patrick Stach's \texttt{md4coll.c}
program~\cite{md4coll} implements Wang's approach and can find MD4 collisions in few seconds
on my Thinkpad~x220 setup specified in \hyperref[app:setup]{Appendix~\ref{app:setup}}.

Let $n$ denote the digest size, hence the size of the hash value $h(x)$ as number of bits.
Due to the birthday paradox, a collision attack has a generic complexity of $2^{n/2}$
whereas pre-image and second pre-image attacks have generic complexities of $2^n$.

Following results by Wang et al., differential cryptanalysis was shown as
powerful tool for cryptanalysis of hash algorithms. This thesis applies those
ideas to satisfiability approaches.

\begin{table}[bt]
  \begin{center}
    \begin{tabular}{cccc}
      \hline \hline
      \multicolumn{4}{c}{Message 1} \\
      \hline
      4d7a9c83 & \textbf{d6cb927a} & \textbf{29d5a578} & 57a7a5ee \\
      de748a3c & dcc366b3 & b683a020 & 3b2a5d9f \\
      c69d71b3 & f9e99198 & d79f805e & a63bb2e8 \\
      \textbf{45dc8e31} & 97e31fe5 & 2794bf08 & b9e8c3e9 \\
      \hline
      \multicolumn{4}{c}{Message 2} \\
      \hline
      4d7a9c83 & \textbf{56cb927a} & \textbf{b9d5a578} & 57a7a5ee \\
      de748a3c & dcc366b3 & b683a020 & 3b2a5d9f \\
      c69d71b3 & f9e99198 & d79f805e & a63bb2e8 \\
      \textbf{45dd8e31} & 97e31fe5 & 2794bf08 & b9e8c3e9 \\
      \hline
      \multicolumn{4}{c}{Hash value of Message 1 and Message 2} \\
      \hline
      5f5c1a0d & 71b36046 & 1b5435da & 9b0d807a \\
      \hline \hline
    \end{tabular}
    \caption[Hexadecimal values of one MD4 collisions given in paper~\cite{wang2004}]{
      One of two MD4 hash collisions provided in~\cite{wang2004}.
      Values are given in hexadecimal, message words are enumerated from
      left to right, top to bottom. Differences are highlighted in
      bold for illustration purposes. For comparison the first bits
      of Message 1 are \texttt{11000001\dots} and the last bits are
      \texttt{\dots10011101}.
      A message represents one block of 512~bits.
      %The hash value consists of 128~bits.
    }
    \label{tab:wang-md4-collision1}
  \end{center}
\end{table}

\section{Differential cryptanalysis}
\label{sec:intro-diff-cryptanalysis}
%
\index{Hash collision}
\begin{defi}[Hash collisoin]
  Given a hash function $h$,
  a hash collision is a pair $(x, x_2)$ with $x \neq x_2$ such that
  $h(x) = h(x_2)$.

\end{defi}

Differential cryptanalysis is based on the idea to consider two execution states
of hash algorithms for slightly different input messages. We trace those difference
to learn about the propagation of message differences.

Considering some hash function $f$, we look for two input messages $x$ and $x_2$
such that the output values $h(x)$ and $h(x_2)$ correspond yielding a hash collision.



\section{Satisfiability}
\label{sec:intro-sat}
%
\index{Boolean function}
\begin{defi}
  A \emph{boolean function} is a mapping $h: X \to Y$ with $X = \left\{0,1\right\}^n$
  for $n \in \mathbb N_{\geq 1}$ and $Y = \left\{0,1\right\}$.
\end{defi}

The following definition gives three basic boolean functions:

\index{AND}
\begin{defi}
  \emph{AND} is a boolean function mapping $X = \left\{0,1\right\}^2$
  to $1$ if all values of $X$ are $1$.
  \emph{OR} is a boolean function mapping $X = \left\{0,1\right\}^2$
  to $1$ if any value of $X$ is $1$.
  \emph{NOT} is a boolean function mapping $X = \left\{0,1\right\}^1$
  to $1$ if the single value of $X$ is $0$.
  All functions return $0$ in the other case.
\end{defi}

\index{Truth table}
\begin{defi}
  A \emph{truth table} unambiguously defines a boolean function
  by enlisting the evaluated truth value for all possible sets of
  inputs.

  Table~\ref{tab:andornot-truthtables} shows truth tables for
  \boolf{AND}, \boolf{OR} and \boolf{NOT}.
\end{defi}

\begin{figure}[!t]
  \centering
  \subfloat[\boolf{AND}]{%
    \begin{tabular}{cc|c}
      $v_1$ & $v_2$ & $f(v_1, v_2)$ \\
     \hline
      $1$ & $1$ & $1$ \\
      $1$ & $0$ & $0$ \\
      $0$ & $1$ & $0$ \\
      $0$ & $0$ & $0$
    \end{tabular}
  }
  ~
  \subfloat[\boolf{OR}]{%
    \begin{tabular}{cc|c}
      $v_1$ & $v_2$ & $f(v_1, v_2)$ \\
     \hline
      $1$ & $1$ & $1$ \\
      $1$ & $0$ & $1$ \\
      $0$ & $1$ & $1$ \\
      $0$ & $0$ & $0$
    \end{tabular}
  }
  ~
  \raisebox{13.6pt}{%
  \subfloat[\boolf{NOT}]{%
    \begin{tabular}{c|c}
      $v$ & $f(v)$ \\
     \hline
      $1$ & $0$ \\
      $0$ & $1$
    \end{tabular}
  }%
  }%
  \caption{Truth tables for \boolf{AND}, \boolf{OR} and \boolf{NOT}}
  \label{tab:andornot-truthtables}
\end{figure}

In the following we discuss how boolean functions are related
to computation in general and hash algorithms specifically.

\index{Algorithm}
\begin{defi}
  An \emph{algorithm} is a step-wise set of instructions to solve a problem.
\end{defi}

One specific set of algorithms transform a given input to some output according
to some rules. Such algorithms are said to satisfy the \emph{I/O property}.
Hash algorithms satisfy the I/O property.

\begin{theorem}
  Every algorithm can be represented as boolean function.
\end{theorem}

This is trivial to see considering that computers are built from logic gates.
However, in practice physical properties are used to employ storage of values,
which does not exist in the model of boolean algebra. This can be mapped to
our boolean model by constants for stored values and assuming an infinite memory.

\index{Satisfiability}
\index{Assignment}
\index{Model}
\begin{defi}
  A boolean function is \emph{satisfiable} iff there exists at least one
  input $x \in X$ such that $h(x) = 1$.
  Every input $x \in X$ satisfying this property is called \emph{model}.
  Every element of $X$ is called \emph{assignment}.
\end{defi}

In the following we will establish some theory to reach
the following property:

\begin{defi}
  A boolean function is \emph{satisfiable} iff a certain state in two hash
  algorithm instances is achievable.
\end{defi}

The generic complexity of SAT determination is given by $2^n$ for $n$ boolean variables.
%This is closely related to the \cP $\overset{?}{\neq}$ \cNP question.
The corresponding tool to determine satisfiability is defined in the following.

\index{SAT solver}
\begin{defi}
  A \emph{SAT solver} is a tool to determine satisfiability of a boolean function.
  If satisfiability is given, it returns some model.
\end{defi}

\section{Thesis Outline}
\label{sec:intro-outline}
%
This thesis is organized as follows:

\begin{description}
\item[In Chapter~\ref{ch:intro}] we discussed the basic properties and fundamentals
of the tools in discussion including hash functions and SAT solvers.

\item[In Chapter~\ref{ch:dc}] we introduce the MD4 and SHA-256 hash functions and
discuss possible approaches in differential cryptanalysis.

\item[In Chapter~\ref{ch:sat}] we discuss SAT solving and potential
approaches to speed up SAT solvers for cryptographic problems.

\item[In Chapter~\ref{ch:results}] we show results of our work
and discuss its implications.

\item[In Chapter~\ref{ch:summary}] we suggest future work based on our results.
\end{description}
