\renewcommand*\chappic{img/intro.pdf}
\renewcommand*\chapquote{}
\renewcommand*\chapquotesrc{}
%
\chapter{Introduction}
\label{ch:intro}
\section{Overview}
\label{sec:intro-overview}
%
Hash functions are used as cryptographic primitives in many applications and protocols.
They take an arbitrary input message and provide a hash value. Input message and hash value
are considered as byte strings in a particular encoding.
The hash value is of fixed length and satisfies several properties which make it useful
in a variety of applications.

In this thesis we will consider the hash algorithms MD4 and SHA-256.
They use basic arithmetic functions like addition and bit-level functions
such as XOR to transform an input to a hash value. We use a bit vector
as input to this implementation and all operations applied to this bit vector
will be represented as clauses of a SAT problem. Additionally we represent
differential characteristics of hash collisions as SAT problem. If and only if
satisfiability is given, the particular differential state is achievable
using two different inputs leading to the same output. As far as SAT solvers
return an actual model satisfying that state, we get an actual hash collision
which can be verified and visualized.
If the internal state of the hash algorithm is too large, the attack can be
computationally simplified by modelling only a subset of steps of the hash algorithm
or changing the modelled differential path.

Based on experience with these kind of problems with previous non-SAT-based tools
we aim to apply best practices to a satisfiability setting.
We will discuss which SAT techniques lead to best performance characteristics
for our MD4 and SHA-256 testcases.

\section{Thesis Outline}
\label{sec:intro-outline}
%
This thesis is organized as follows:

\begin{description}
\item[In Chapter~\ref{ch:intro}] we briefly introduce basic subjects of this
  thesis. We explain our high-level goal involving hash functions and SAT solvers.

\item[In Chapter~\ref{ch:hash}] we introduce the MD4 and SHA-256 hash functions.
  Certain design decisions imply certain properties which can be used in differential
  cryptanalysis. We discuss those decisions in this chapter after a formal definition
  of the function itself. Beginning with this chapter we develop a theoretical notion
  of our tools.

\item[In Chapter~\ref{ch:dc}] we discuss approaches of differential cryptanalysis.
  We start off with work done by Wang, et al. and followingly introduce differential
  notation to simplify representation of differential states. This way we can easily dump
  hash collisions.

\item[In Chapter~\ref{ch:sat}] we discuss SAT solving. We give a brief overview
  over used SAT solvers and discuss how we can speed up SAT solvers for cryptographic
  problems.

\item[In Chapter~\ref{ch:features}] we define SAT features which help us to
  classify SAT problems. This is a small subproject we did to look at properties
  of resulting DIMACS~CNF files.

\item[In Chapter~\ref{ch:enc}] we discuss how we represent a problem (i.e. the hash
  function and a differential characteristic) as SAT~problem. This ultimatively
  allows us to solve the problem using a SAT solver.

\item[In Chapter~\ref{ch:results}] we show data as result of our work.
  Runtimes are the main part of this chapter, but also results of the SAT~features
  project are presented.

\item[In Chapter~\ref{ch:summary}] we suggest future work based on our results.
\end{description}
