\renewcommand*\chappic{}
\renewcommand*\chapquote{}
\chapter{Summary and Future Work}
\label{ch:summary}
%
\section{Related work}
\label{sec:results-related}
%
In my bachelor thesis~\cite{bach} we tried to integrate a SAT solver into
our department's existing tool which encodes propagation explicitly. This approach
was not very successful as restarts between hash algorithm rounds implied that
intermediate results by the SAT solver got lost. This was our motivation
for this master thesis: We represent all details in CNF.

Research was already done by Ilya Mironov and Lintao Zhang~\cite{mironov2006applications}
to apply SAT solvers to differential cryptanalysis specifically to find hash collision
in MD4. However whereas their basic approach seems to correspond to our approach described
in Section~\ref{sec:enc-original}, our implementation uses additional SAT design tweaks
to improve our results. Also because 10 years have gone since publication, SAT technology
has progressed and modern SAT solvers on modern hardware provide better results.

\section{Conclusion}
\label{sec:conclusion}
%
We successfully found full-round hash collisions for MD4
using SAT solvers mentioned in Section~\ref{sec:sat-solvers}.
We applied tweaks to the lingeling SAT solver to improve our
runtime results further and found 24-round hash collisions
for SHA-256. Our attack starting points for MD4 ---
Testcases~\ref{fig:tcA}, \ref{fig:tcB} and \ref{fig:tcC} ---
were found by Yusuke Naito, Yu Sasaki, Noboru Kunihiro and
Kazuo Ohta~\cite{sasaki2007new}. Our starting points for SHA-256
--- Testcases~\ref{fig:tc18}, \ref{fig:tc21}, \ref{fig:tc23}
and \ref{fig:tc24} --- were found by Ivica Nikoli{\'c} and
Alex Biryukov~\cite{nikolic2008collisions}.

\section{Future work}
\label{sec:future}
%
Future work might want to consider our design decisions made in
chapter~\ref{ch:enc}.

In general, it would be interesting to generate our testcase for a broader
set of differential characteristics. Probably we can come up with empirical
results showing a relation between the size of unspecified areas in the
differential characteristics and the evaluated runtime.

Furthermore, many SAT-related effort could be put to thoroughly discuss
why differential description provides such a dramatic performance improvement.
This necessarily means the SAT solver is generally not capable of deriving
the useful clauses, differential description provides. The resulting
numbers of restarts could also be subject of further research.

As far as cnf-analysis is concerned, the project aims to extend to
a larger set of SAT features. Furthermore benchmarks should be aggregated
from more sources, but Armin Biere pointed out in private conversation
that not all benchmarks of SAT competitions become publicly available.
This avoids machine learning hacks where submitted SAT solvers perform
exceptionally well on known benchmarks. Our main contribution is a search
interface to search for SAT features given the cnfhash of a CNF file.
We hope to get in touch with new SAT feature ideas and SAT benchmark files.

\section{Contributions}
\label{sec:contributions}
%
To strengthen Reproducible Research, the source code and data resulting from this thesis is available online.
It allows the reader to run the experiments again and verify our claims.
We did our best to describe our hardware setup as accurately as possible.
At the following website, any results part of this project are collected:

\begin{quote}
  \url{http://lukas-prokop.at/proj/megosat/}
\end{quote}

Several subprojects are part of this master thesis:
\begin{description}
\item[algotocnf]\hfill{} \\
  A python library implementing the encoding described in Chapter~\ref{ch:enc}. \\[4pt]
  \textbf{Python3 library and program:} \url{https://github.com/prokls/algotocnf}
\item[cnf-hash]\hfill{} \\
  A standardized way to produce a unique hash for CNF files \\[4pt]
  \textbf{Go implementation:} \url{https://github.com/prokls/cnf-hash-go} \\
  \textbf{Python3 implementation:} \url{https://github.com/prokls/cnf-hash-py} \\
  \textbf{Testsuite:} \url{https://github.com/prokls/cnf-hash-tests2}
\item[cnf-analysis]\hfill{} \\
  Evaluate SAT features for a given CNF file. \\[4pt]
  \textbf{Go implementation:} \url{https://github.com/prokls/cnf-analysis-go} \\
  \textbf{Python3 implementation:} \url{https://github.com/prokls/cnf-analysis-py} \\
  \textbf{Testsuite:} \url{https://github.com/prokls/cnf-analysis-tests}
\end{description}
