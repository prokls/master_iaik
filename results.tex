\renewcommand*\chappic{img/megosat.pdf}
\renewcommand*\chapquote{}
\renewcommand*\chapquotesrc{}
\chapter{Results}
\label{ch:results}
%
\section{Evaluating SAT features}
\label{sec:results-features}

TODO: do cryptoproblems distinguish from other problems?

TODO: do our benchmark distinguish from other problems?

TODO: answer the 8 questions posed previously

\section{Finding hash collisions}
\label{sec:results-attacks}

TODO: is simplification worth it?

TODO: discuss runtime and development, tuning by doing difference variables first,
diff desc makes a difference

Appendix~\ref{app:runtimes} provide a more exhaustive list of runtimes retrieved.

\subsection{Attacking MD4}
\label{sec:results-md4}
%
In Section~\ref{sec:enc-algotocnf} we introduced a basic encoding involving two hash algorithm
instances and difference variables. Constraints resulting from the hash algorithm description
and given differential characteristic are added.

We considered MD4 testcases~A, B and C (compare with Appendices~\ref{app:tcA}, \ref{app:tcB} and \ref{app:tcC})
and generated the corresponding CNF files. The SAT solvers mentioned in Section~\ref{sec:sat-solvers}
were used to evaluate whether the problem is solvable in reasonably time. For every testcase we
defined a time limit of at most 1 day (i.e. 86,400 seconds). Some testcases listed have been evaluated
for a larger time limit.

\begin{table}[!h]
  \begin{center}
    \begin{tabular}{lcc}
      SAT solver                & testcase      & runtime (in seconds) \\
    \hline
      minisat~2.2.0             & MD4, A        & 65 \\
      cryptominisat~4.5.3       & MD4, A        & 24 \\
      cryptominisat~5.0.0       & MD4, A        & 29 \\
      glucose~4.0               & MD4, A        & 10 \\
      glucose-syrup~4.0         & MD4, A        & 31 \\
      lingeling-ats1            & MD4, A        & TODO \\
      lingeling-ats1o1          & MD4, A        & 18 \\
      lingeling-ats1o2          & MD4, A        & TODO \\
      lingeling-ats1o4          & MD4, A        & 125745 \\
      plingeling-ats1o1         & MD4, A        & 88 \\
      treeneling-ats1o1         & MD4, A        & 64
    \end{tabular}
  \end{center}
\end{table}

\subsection{Improvements with differential description}
\label{sec:result-diff-desc}

\subsection{Modifying the guessing strategy}
\label{sec:results-guessing}


\section{Related work}
\label{sec:results-related}

\section{Conclusion}
\label{sec:conclusion}

TODO: we attacked MD4 and SHA2, but can see the problems with

\section{Contributions}
\label{sec:contributions}
%
To strengthen Reproducible Research, the source code and data resulting from this thesis is available online.
It allows the reader to run the experiments again and verify our claims.
We did our best to describe our hardware setup as accurately as possible.
At the following website, any results part of this project are collected:

\begin{quote}
  \url{http://lukas-prokop.at/proj/megosat/}
\end{quote}

Several subprojects are part of this master thesis:
\begin{description}
\item[algotocnf]\hfill{} \\
  A python library implementing the encoding described in chapter~\ref{ch:enc}. \\[4pt]
  \textbf{Python3 library and program:} \url{https://github.com/prokls/algotocnf}
\item[cnf-hash]\hfill{} \\
  A standardized way to produce a unique hash for CNF files \\[4pt]
  \textbf{Go implementation:} \url{https://github.com/prokls/cnf-hash-go} \\
  \textbf{Python3 implementation:} \url{https://github.com/prokls/cnf-hash-py} \\
  \textbf{Testsuite:} \url{https://github.com/prokls/cnf-hash-tests2}
\item[cnf-analysis]\hfill{} \\
  Evaluate SAT features for a given CNF file. \\[4pt]
  \textbf{Go implementation:} \url{https://github.com/prokls/cnf-analysis-go} \\
  \textbf{Python3 implementation:} \url{https://github.com/prokls/cnf-analysis-py} \\
  \textbf{Testsuite:} \url{https://github.com/prokls/cnf-analysis-tests}
\end{description}
