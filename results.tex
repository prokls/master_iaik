\renewcommand*\chappic{img/megosat.pdf}
\renewcommand*\chapquote{}
\renewcommand*\chapquotesrc{}
\chapter{Results}
\label{ch:results}
%
In Section~\ref{ch:features} we looked at SAT features which to some extent characterize
a SAT problem. In Section~\ref{ch:enc} we discussed problem encoding for our hash collision
attacks. In this section, we want to look at the feature value results and runtime results
we retrieved.

\section{Evaluating SAT features}
\label{sec:results-features}

In the introduction of chapter~\ref{ch:features} we posed 8 questions.
In the following we want to answer them with the data provided by the
cnf-analysis project.

\begin{description}
\item[Given an arbitrary literal. What is the percentage it is positive?]
  We look at every clause and determine the ratio of positive to the total number of literals.
  We determine the mean per CNF file and the mean among all CNF files
  and retrieve a value of $0.48$ meaning that 48~\% of the literals are positive.
\item[What is the variables / clauses ratio?]
  In average a CNF file has 12,219 variables and 89,541 clauses.
  Its variables-clauses ratio is 0.1365.
\item[How many literals occur only once either positive or negative?]
  In average there are 36 existential literals per CNF file,
  but its standard deviation of 967 is very high.
\item[What is the average and longest clause length among CNF benchmarks?]
  The average clause length is 3.04 with a standard deviation of 0.99
  and the longest clause length found was 61,473. Long clauses are typically
  outliers excluding specific assignments.
\item[How many Horn clauses exist in a CNF?]
  In average 29,994 goal clauses and 31,315 definite clauses exist
  with an average number of 83649 clauses in a CNF file.
\item[Are there any tautological clauses?]
  In a file, 1679 tautological literals have been found. However,
  its mean is 0.07 with a standard deviation of $9.63$ meaning that tautological
  clauses are very rare.
\item[Are there any CNF files with more than one connected variable component?]
  Indeed, an average CNF file contains 67.07 connected variable components.
  However, its median is 1 anyway implying that at least half of the CNF files
  have only 1 connected variable component.
\item[How many variables of a CNF are covered by unit clauses?]
  In average 124 variables are covered by unit clauses. This is an insignificant
  number compared to 12,219 variables in an average CNF.
\end{description}

Do our generated problems distinguish from these average problems?

\begin{itemize}
  \item In average our (MD4 and SHA-256 hash collision) problems contain
        84,821 variables and 515,646 clauses meaning that our problems
        are comparably large.
% TODO
%  \item Due to the cryptographic high diffusion we expected a 
\end{itemize}

%TODO: do cryptoproblems distinguish from other problems?

%TODO: do our benchmark distinguish from other problems?

\section{Finding hash collisions}
\label{sec:results-attacks}

TODO: is simplification worth it?

TODO: discuss runtime and development, tuning by doing difference variables first,
diff desc makes a difference

Appendix~\ref{app:runtimes} provide a more exhaustive list of runtimes retrieved.

\subsection{Attacking MD4}
\label{sec:results-md4}
%
In Section~\ref{sec:enc-algotocnf} we introduced a basic encoding involving two hash algorithm
instances and difference variables. Constraints resulting from the hash algorithm description
and given differential characteristic are added.

We considered MD4 testcases~A, B and C (compare with Appendices~\ref{app:tcA}, \ref{app:tcB} and \ref{app:tcC})
and generated the corresponding CNF files. The SAT solvers mentioned in Section~\ref{sec:sat-solvers}
were used to evaluate whether the problem is solvable in reasonably time. For every testcase we
defined a time limit of at most 1 day (i.e. 86,400 seconds). Some testcases listed have been evaluated
for a larger time limit.

\begin{table}[!h]
  \begin{center}
    \begin{tabular}{lcc}
      SAT solver                & testcase      & runtime (in seconds) \\
    \hline
      minisat~2.2.0             & MD4, A        & 65 \\
      cryptominisat~4.5.3       & MD4, A        & 24 \\
      cryptominisat~5.0.0       & MD4, A        & 29 \\
      glucose~4.0               & MD4, A        & 10 \\
      glucose-syrup~4.0         & MD4, A        & 31 \\
      lingeling-ats1            & MD4, A        & TODO \\
      lingeling-ats1o1          & MD4, A        & 18 \\
      lingeling-ats1o2          & MD4, A        & TODO \\
      lingeling-ats1o4          & MD4, A        & 125745 \\
      plingeling-ats1o1         & MD4, A        & 88 \\
      treeneling-ats1o1         & MD4, A        & 64
    \end{tabular}
  \end{center}
\end{table}

\subsection{Improvements with differential description}
\label{sec:result-diff-desc}

\subsection{Modifying the guessing strategy}
\label{sec:results-guessing}


\section{Related work}
\label{sec:results-related}

\section{Conclusion}
\label{sec:conclusion}

TODO: we attacked MD4 and SHA2, but can see the problems with

\section{Contributions}
\label{sec:contributions}
%
To strengthen Reproducible Research, the source code and data resulting from this thesis is available online.
It allows the reader to run the experiments again and verify our claims.
We did our best to describe our hardware setup as accurately as possible.
At the following website, any results part of this project are collected:

\begin{quote}
  \url{http://lukas-prokop.at/proj/megosat/}
\end{quote}

Several subprojects are part of this master thesis:
\begin{description}
\item[algotocnf]\hfill{} \\
  A python library implementing the encoding described in chapter~\ref{ch:enc}. \\[4pt]
  \textbf{Python3 library and program:} \url{https://github.com/prokls/algotocnf}
\item[cnf-hash]\hfill{} \\
  A standardized way to produce a unique hash for CNF files \\[4pt]
  \textbf{Go implementation:} \url{https://github.com/prokls/cnf-hash-go} \\
  \textbf{Python3 implementation:} \url{https://github.com/prokls/cnf-hash-py} \\
  \textbf{Testsuite:} \url{https://github.com/prokls/cnf-hash-tests2}
\item[cnf-analysis]\hfill{} \\
  Evaluate SAT features for a given CNF file. \\[4pt]
  \textbf{Go implementation:} \url{https://github.com/prokls/cnf-analysis-go} \\
  \textbf{Python3 implementation:} \url{https://github.com/prokls/cnf-analysis-py} \\
  \textbf{Testsuite:} \url{https://github.com/prokls/cnf-analysis-tests}
\end{description}
