\renewcommand*\chappic{intro.pdf_tex}
\chapter{Introduction}
\label{ch:introduction}
%
Hash functions are used as cryptographic primitives in many applications and protocols.
They take an arbitrary input message and compute a hash value. Input message and hash value
are considered as byte strings in a particular encoding.
The hash value is of fixed length and satisfies several properties which make them useful
to enable data and origin integrity.

\begin{todo}
  explain our attack approach using terminology used in the following
\end{todo}

\section{Preliminaries}
\label{sec:intro-prelim}

\index{Hash value}
\index{Preimage}
\index{Hash function}
\begin{defi}[Hash function]
  A \emph{hash function} is a mapping $h: X \to Y$ with $X = \left\{0,1\right\}^*$ and
  $Y = \left\{0,1\right\}^n$ for $n \in \mathbb N_{\geq 1}$.
  \begin{itemize}\itemsep0pt
    \item Let $x \in X$. $h(x)$ is called \emph{hash value of $x$}.
    \item Let $h(x) = y \in Y$. $y$ is called \emph{preimage of $y$}.
  \end{itemize}
\end{defi}

One example showing the use of hash functions as primitives is PKCS~\#5
specified in RFC~2898~\cite{rfc2898}. Section~5.2 specifies PBKDF1 and PBKDF2
using an arbitrary pseudorandom function to derive password-based keys.
Hash algorithms can be used as those pseudorandom functions.
Given a minimum iteration count of $1000$, as defined in section~4.2,
yields the additional requirement that fast computation of hash values
for given $x \in X$ is desirable.

A hash function has to satisfy the following security requirements:

\index{Preimage resistance}
\begin{defi}[Preimage resistance]
  Given $y \in Y$.
  A hash function $h$ is preimage resistant iff it is computationally infeasible
  to find $x \in X$ such that $h(x) = y$.
\end{defi}

\index{Second-preimage resistance}
\begin{defi}[Second-preimage resistance]
  Given $x \in X$.
  A hash function $h$ is second-preimage resistant iff it is computationally infeasible
  to find $x_2 \in X$ with $x \neq x_2$ such that $h(x) = h(x_2)$.
  $x_2$ is called \emph{second preimage}.
\end{defi}

\index{Collision resistance}
\begin{defi}[Collision resistance]
  A hash function $h$ is collision resistant iff it is computationally infeasible to
  find any two $x \in X$ and $x_2 \in X$ with $x \neq x_2$ such that $h(x) = h(x_2)$.
\end{defi}

As far as hash functions accept input strings of arbitrary length, but return a fixed
size output string, existence of preimages and collisions is unavoidable~\cite{schlaffer}.
However, good hash functions make it very difficult to determine collisions or preimages.

\section{Cryptanalysis of Hash Functions}
\label{sec:intro-cryptanalysis}
%
In August 2004, Wang et al. published results at Crypto'04~\cite{wang2004} which revealed
that MD4, MD5, HAVAL-128 and RIPEMD can be broken practically using differential cryptanalysis.
On an IBM~P690 machine, an MD5 collision can be computed in about one hour using this approach.
Collisions for HAVAL-128, MD4 and RIPEMD were found as well. Patrick Stach's \texttt{md4coll.c}
program~\cite{md4coll} implements Wang's approach and can find MD4 collisions in few seconds
on my Thinkpad x220 machine powering a .

Due to the birthday paradox, a collision attack has a generic complexity of $2^{\frac n2}$ whereas pre-image and second pre-image attacks have a generic complexity of $2^n$.

\begin{todo}
  discuss their approach in more detail, introduce basics of diff cryptanalysis (w/o notation)
\end{todo}

\section{Satisfiability}
\label{sec:intro-sat}
%
\index{Boolean function}
\begin{defi}
  A \emph{boolean function} is a mapping $h: X \to Y$ with $X = \left\{0,1\right\}^n$
  for $n \in \mathbb N_{\geq 1}$ and $Y = \left\{0,1\right\}$.
\end{defi}

The following definition gives three basic boolean functions:

\index{AND}
\begin{defi}
  \emph{AND} is a boolean function mapping $X = \left\{0,1\right\}^2$
  to $1$ if all values of $X$ are $1$.
  \emph{OR} is a boolean function mapping $X = \left\{0,1\right\}^2$
  to $1$ if any value of $X$ is $1$.
  \emph{NOT} is a boolean function mapping $X = \left\{0,1\right\}^1$
  to $1$ if the single value of $X$ is $0$.
  All functions return $0$ in the other case.
\end{defi}

\index{Truth table}
\begin{defi}
  A \emph{truth table} unambiguously defines a boolean function
  by enlisting the evaluated truth value for all possible sets of
  inputs.

  Table~\ref{tab:andornot-truthtables} shows truth tables for
  \boolf{AND}, \boolf{OR} and \boolf{NOT}.
\end{defi}

\begin{figure}[!t]
  \centering
  \subfloat[\boolf{AND}]{%
    \begin{tabular}{cc|c}
      $v_1$ & $v_2$ & $f(v_1, v_2)$ \\
     \hline
      $1$ & $1$ & $1$ \\
      $1$ & $0$ & $0$ \\
      $0$ & $1$ & $0$ \\
      $0$ & $0$ & $0$
    \end{tabular}
  }
  ~
  \subfloat[\boolf{OR}]{%
    \begin{tabular}{cc|c}
      $v_1$ & $v_2$ & $f(v_1, v_2)$ \\
     \hline
      $1$ & $1$ & $1$ \\
      $1$ & $0$ & $1$ \\
      $0$ & $1$ & $1$ \\
      $0$ & $0$ & $0$
    \end{tabular}
  }
  ~
  \subfloat[\boolf{NOT}]{%
    \begin{tabular}{c|c}
      $v$ & $f(v)$ \\
     \hline
      $1$ & $0$ \\
      $0$ & $1$
    \end{tabular}
  }%
  \caption{Truth tables for \boolf{AND}, \boolf{OR} and \boolf{NOT}}
  \label{tab:andornot-truthtables}
\end{figure}

In the following we discuss how boolean functions are related
to general computations:

\index{Algorithm}
\begin{defi}
  An \emph{algorithm} is a step-wise set of instructions to solve a problem.
  An algorithm transforms a given input to some output.
  Turing machines can be utilized as computational model which is conjectured
  to be equivalent to Lambda Calculus.
\end{defi}

\begin{todo}
  go on
\end{todo}

\begin{theorem}
  Every algorithm can be represented as boolean function.
\end{theorem}

\begin{todo}
  go on
\end{todo}



\section{Satisfiability}
\label{sec:into-sat}
%
\begin{todo}
  intro
\end{todo}


\index{Satisfiability}
\begin{defi}
  A boolean function is \emph{satisfiable} iff there exists at least one
  input $x \in X$ such that $h(x) = 1$.
\end{defi}



\section{Thesis Outline}
%
This thesis is organized as follows:

\begin{description}
\item[In Chapter~\ref{ch:introduction}] we discussed the basic properties and fundamentals
of the tools in discussion including hash functions and SAT solvers.

\item[In Chapter~\ref{ch:diff-crypt}] we introduce the MD4 hash function and
discuss possible approaches in differential cryptanalysis.

\item[In Chapter~\ref{ch:sat-solving}] we discuss SAT solving and potential
approaches to speed up SAT solvers for cryptographic problems.

\item[In Chapter~\ref{ch:results}] we show results of our work
and discuss its implications.
\end{description}

%Some basic background, especially related work and approaches
