\renewcommand*\chappic{img/encoding.pdf}
\renewcommand*\chapquote{There is concensus that encoding techniques usually have a dramatic impact on the efficiency of the SAT solver}
\renewcommand*\chapquotesrc{Magnus Bj\"ork}
\chapter{Problem encoding}
\label{ch:enc}

In Chapter~\ref{ch:sat}, we already discussed how SAT solvers
work and which input they take. We also sketched how hash algorithm
properties got broken using differential cryptanalysis
(Chapter~\ref{ch:dc}). In this chapter, we combine those
subjects and describe how we run SAT solvers to find hash collisions.

We developed a basic prototype with the STP SMT solver. In the following,
we wanted to tweak the CNF used by the SAT solver and wrote
our own library \emph{algotocnf} to generate CNFs modelling
variable differences and their logic; as illustrated in Section~\ref{ch:dc}.
In the referred section, we distinguished 5 different approach. We evaluate
the performance of those approaches in Chapter~\ref{ch:results}.

Every section represents a major approach whereas subsections
represent derivatives of this approach with minor changes.

\section{Basic approach}
\label{sec:enc-stp}
%
Our first approach started with Simple Theorem Prover (STP)~\cite{stp}
initially written by Vijay Ganesh and David L. Dill.
It is currently maintained by Mate Soos.

STP is an SMT solver which allows to declare bitvectors. A bitvector is an array of
Boolean variables providing high-level constructs such as additions or right-shift
through an interface. Writing all clauses individually to model a hash algorithm
is too cumbersome to be done in practice and STP simplifies this process.
STP is recommendable as a tool to model arithmetic and bitwise functions.

First we wrote an implementation using the CVC language to model the MD4 hash algorithm.
We provide a bitvector to the hash algorithm instance. When applying the corresponding
bitwise operations we generate expressions such as \texttt{ASSERT(y = 0bin00000101)}
to model the assignment of a constant. Here the desired constant is assigned variable
\texttt{y}, because of equivalence. \texttt{y} is required to have the constant
after this expression as value. Whereas we generate the \texttt{ASSERT} statement,
it is STP's task to generate the CNF formula and solve it with a SAT solver.

We take two hash algorithm instances and an additional bitvector \texttt{diff}
for every pair of bitvectors (\texttt{bv1}, \texttt{bv2}) where \texttt{bv2} represents
the corresponding bitvector of the second hash algorithm instance to \texttt{bv1}.
We claim \texttt{ASSERT(diff = BVXOR(bv1, bv2))} to ensure that \texttt{diff}
represents the difference between \texttt{bv1} and \texttt{bv2}. Given the bitconditions
for a bitvector from a differential characteristic, we require \texttt{diff} to enforce
those particular differences. This corresponds to the idea of differential cryptanalysis
introduced in Chapter~\ref{ch:dc}.

It is now trivial to consider a differential characteristic of MD4 such as Testcase~A (see
Section~\ref{sec:tcA}), where all differences are set, but the individual values in both
instances need to be assigned. We generated the corresponding CVC~input for STP.
STP solves this particular problem within a second. Testcase~B (compare with
Section~\ref{sec:tcB}) already provides a more complex example taking
40 minutes to solve, because not all differences are set.
We used minisat as SAT solver in the backend, even though STP
allows to replace it for CryptoMiniSat which is a more modern and versatile SAT solver.

Even though STP allows to come to useful results pretty quickly,
it seems cumbersome to model all hash algorithms in the CVC language.
STP provides a python interface meaning that pure Python implementations
of hash algorithms can be taken with little modifications to model the
hash algorithm itself. We add code to declare the difference bitvectors
\texttt{diff} and finally add the constraints resulting from the differential
characteristic.

This interface switch introduces no significant performance difference.

As a next step, we wanted to improve the evaluation performance to tackle more difficult
problems such as SHA-256. We considered this design as a working prototype of a basic approach to be
improved upon.

STP seems not suited for our next goal, because we wanted to modify
the particular CNF generated for the SAT solver and needed good control
over the SAT encoding which we expected to have a major influence
on the performance.

\section{algotocnf}
\label{sec:enc-algotocnf}
%
We implemented our own library \emph{algotocnf} to achieve greater
flexibility in our SAT encoding.

\subsection{Two instances and its difference}
\label{sec:enc-original}
%
\index{algotocnf}
Similar to STP, algotocnf generates a CNF for a given hash algorithm implementation.
Besides modelling bitvectors, it also implements differential bitvectors which
inherently handle the difference bitvectors \texttt{diff} which contain difference
variables. It can also directly takes differential characteristics (such as
Table~\ref{tab:wang-collision-propagated}, specified in Chapter~\ref{ch:dc})
as input. Similarly to STP, it implements arithmetic and bitwise operations.

We think \emph{algotocnf} mainly differs from other SMT tools like STP,
because of its implementation of differential logic.

To model our CNF \emph{algotocnf} implements the following strategy:
\begin{enumerate}
  \item Take a differential characteristic and the hash algorithm as input.
  \item Every bit gets represented as a Boolean variable.
    If you apply addition, operator overloading in python will ensure
    that clauses are generated to describe the addition consisting of
    \boolf{XOR}s and \boolf{MAJ}s. Every operation is modelled as assignment.
    Hence, an operation using a few Boolean variables is equivalent
    to a single variable which represents the result.
    Similarly other operations related to integers are implemented as well.
  \item Constants used in the implementation are automatically converted
    to bitvectors with unit clauses.
  \item After running the hash algorithm with bitvectors per instance,
    all constraints related to the hash algorithm are added.
  \item Afterwards, the differential characteristic is read. Values such as $A_i$
    represent intermediate states of bitvectors. Therefore the corresponding bitvectors
    are looked up and equivalences with temporary bitvectors are added.
    Those temporary bitvectors are initialized with all constraints resulting from
    the bit conditions of this bitvector (please refer to Chapters~\ref{ch:hash}
    and \ref{ch:dc} for details).
    In conclusion, all constraints resulting from the differential characteristic are added.
  \item Finally, the SAT solver is called. The CNF was mostly solved on a cluster
    specified in Appendix~\ref{app:setup}.
  \item Afterwards the program is run again to
    create the exact same problem instance and the solver's solution replaces
    symbolic values with actual Boolean values. The resulting differential
    characteristic is parsed backed and printed as differential characteristic
    where all bits have been determined (i.e. a hash collision has been found).
\end{enumerate}

When adding clauses resulting from
the differential characteristic as constraints, the question arises how those
bit conditions are encoded. Essentially, we have only Boolean values available,
but bit conditions tell constraints such as \enquote{a difference is given,
but the actual value is unknown}.

It seemed trivial to add a \emph{difference variable} for every pair of Boolean
values representing a bit in the two instances. Furthermore, the difference
variable $\Delta x$ is connected by a \boolf{XOR} with the variables of the pair $(x, x')$.
\[ \Delta x = x \oplus x' \]
Therefore, it is trivial for a preprocessor to simplify the formula
appropriately. Hence, we don't expect runtime differences for the larger
amount of variables.

And finally we expect the CNF to inherit a property of hash functions.
Inputs are provided into the hash algorithm and strongly intermingled
with other values. This results in a high diffusion and almost every
variable is expected to share a clause with another variable.

The difference variables design corresponds to \texttt{diff} bit vectors
in the STP and therefore models the difference variables described
in Chapter~\ref{ch:dc}.
The design decisions of this encoding are fundamental to the resulting
runtime as discussed in Chapter~\ref{ch:results}.

\subsection{Adding the differential description}
\label{sec:enc-diff-desc}
%
Using the approach in the previous section, we were able to find actual MD4 collisions
using a SAT solver (please refer to Section~\ref{sec:results-md4}).
We used a reused our implementation of MD4 for SHA-256 and replaced the hash algorithm
implementation. This implementation obviously presented worse
runtime results, because the internal state of SHA-256 is much larger
(by a factor of at least 2). Can we further improve the runtime of the SAT solver?

Since we work with bitvectors and apply high-level operations like \boolf{MAJ} or addition,
we can additionally implement how differences in those operations propagate.
Magnus Daum's thesis on \enquote{Cryptanalysis of Hash Functions of the
MD4-Family}~\cite[Table 4.4]{daum2005cryptanalysis} discusses how differences propagate in
Boolean functions. Trivially, \boolf{XOR}s propagate differences the way they are\footnote{
A difference in the arguments of two \boolf{XOR} instances remains the same difference
after applying \boolf{XOR} to each instance}. Another example is \boolf{IF}:
Let $a$, $b$ and $c$ be difference variables and \boolf{IF} is applied to both corresponding
hash algorithm instances. $r$ is the difference variable of the result.
Then its differential behavior states that
\[ (0, 1, 1) \implies 1 \qquad (0, 0, 0) \implies 0 \]
where $(\text{IN}) \implies \text{OUT}$ denotes an input-output relation
and in all other cases, the difference can be either 1 or 0.
Because of this behavior we add clauses to explicitly describe this behavior:
\[ (\neg a \land b \land c) \implies r \quad \iff \quad a \lor \neg b \lor \neg c \lor r \]
We also model the second behavior:
\[ (\neg a \land \neg b \land \neg c) \implies \neg r \quad \iff \quad a \lor b \lor c \lor \neg r \]

This approach explicitly models differentiable behavior, which should be deducible
by the SAT solver itself based on the clauses we added before.
However, this lead to a major speedup which can be observed in the runtime results
of Chapter~\ref{ch:results}.

\subsection{Difference variables first}
\label{sec:enc-diff-desc-ocnf}
%
In this approach we reduce the number of evaluated differences by guessing
Boolean value false first for difference variables.

\begin{prop}
  Deriving difference values first, followed by actual bit values for the two instances,
  leads to a speedup.
\end{prop}

This proposed principle is fundamental to differential cryptanalysis. A previous tool
at IAIK (TU Graz) implements propagation of hash algorithm values without a SAT solver
and this strategy is essential to good performance. This strategy was introduced
in the very early days of differential cryptanalysis and was also used by Wang et
al.~\cite{wang2004} to find their hash collisions.

For our SAT solver, we want to establish the following strategy:
Take some CNF which includes difference variables. We assign a Boolean value
to every difference variable unless a contradiction is found. For all the
remaining variables we try to find a satisfiable assignment. If none can be found,
we consecutively toggle the Boolean value of difference variables to cover all
possible assignments and find a satisfiable one. 

It is important to point out that DIMACS does not specify a way to annotate Boolean
variables. As such that SAT solver cannot distinguish between difference variables
and variables of the instances. Therefore, implementing this approach requires a custom
SAT solver which is given with lingeling~ats1o1.

Another claim is important for this approach:

\begin{prop}
  \label{prop:false-first}
  Guessing difference values false first, followed by true,
  should solve hash collision problems faster.
\end{prop}

This claim is justified by the desire to find sparse characteristics with few differences
in intermediate variables to increase the probability of values cancelling each other out
in the later rounds.

\subsection{A lightweight approach}
\label{sec:enc-lightweight}
%
In this approach we made a step back and considered the ideas of the previous
section, but neglected the differential description. This approach was interesting
to quantify the effect introduced by adding the differential description.

\subsection{Influencing the evaluation order}
\label{sec:enc-diff-desc-eo}
%
To take the idea to influence the evaluation order to the next level,
we enforced the evaluation order even stronger by applying the following SAT~design:

Let $\Delta x$ be the difference variable of pair $(x, x')$. We introduce a new Boolean
variable $x^*$ called \emph{preference variable}. We add clause
\[ x^* = (\Delta x \land x) \]
and explicitly tell the SAT solver to guess on $x^*$ before guessing on $\Delta x$, $x$ or $x'$.

The SAT solver will assign $x'=0$ first, because of the evaluation order. So either $\Delta x$
or $x$ must be false. $\Delta x$ is assigned false, because as difference variable it has a higher
priority over $x$. Equivalently for $x'=1$, $\Delta x$ needs to be true. So we actually
achieve an early guess on the difference variable.

In Chapter~\ref{ch:results} we evaluated the performance of this approach.
