\renewcommand*\chappic{img/intro.pdf}
\renewcommand*\chapquote{}
\renewcommand*\chapquotesrc{}
%
\chapter{Introduction}
\label{ch:intro}
\section{Overview}
\label{sec:intro-overview}
%
Hash functions are used as cryptographic primitives in many applications and protocols.
They take an arbitrary input message and provide a hash value. Input message and hash value
are considered as byte strings in a particular encoding.
The hash value is of fixed length and satisfies several properties which make it useful
in a variety of applications.

In this thesis we will consider the hash algorithms MD4 and SHA-256 and represent
differential characteristics of hash collisions as SAT problem. If and only if
satisfiability is given, the particular differential state is achievable
using two different inputs leading to the same output. As far as SAT solvers
return an actual model satisfying that state, we get an actual hash collision
which can be verified and visualized.
If the internal state of the hash algorithm is too large, the attack can be
computationally simplified by modelling only a subset of steps of the hash algorithm
or changing the modelled differential path.

Based on experience with these kind of problems with previous non-SAT-based tools
we aim to apply best practices to a satisfiability setting.
We will discuss which SAT techniques lead to best performance characteristics
for our MD4 and SHA-256 testcases.

\section{Thesis Outline}
\label{sec:intro-outline}
%
This thesis is organized as follows:

\begin{description}
\item[In Chapter~\ref{ch:intro}] we discussed the basic properties and fundamentals
of the tools in discussion including hash functions and SAT solvers.

\item[In Chapter~\ref{ch:dc}] we introduce the MD4 and SHA-256 hash functions and
discuss possible approaches in differential cryptanalysis.

\item[In Chapter~\ref{ch:sat}] we discuss SAT solving and potential
approaches to speed up SAT solvers for cryptographic problems.

\item[In Chapter~\ref{ch:results}] we show results of our work
and discuss its implications.

\item[In Chapter~\ref{ch:summary}] we suggest future work based on our results.
\end{description}
