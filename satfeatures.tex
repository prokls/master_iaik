\chapter{SAT features}
\label{ch:sat-features}


\section{SAT features and CNF analysis}
\label{sec:sat-features}
%
At the very beginning I was very intrigued with the question
\enquote{What is an \enquote{average} SAT problem?}. Answers to this
question can help to optimize SAT solver memory layouts.
But originally I was wondering whether our differential cryptanalysis
SAT problems distinguish from \enquote{average} SAT problems in some
very basic properties. \\
First of all, we need to elaborate on the question itself.

\index{SAT feature}
\index{Feature value}
\begin{defi}[SAT feature]
  A \emph{SAT feature} is a statistical value (named \emph{feature value})
  retrievable from some given SAT problem in some well-defined encoding.

  A SAT feature is called \emph{performance-driven} if the runtime of any
  computation contributes to the feature value.
\end{defi}

The most basic example of a SAT feature is the number of variables and clauses
of a given SAT problem. This SAT feature is stored in the CNF header of a SAT
problem encoded in the DIMACS format.

It should be computationally easy to evaluate SAT features of a given SAT problem.
The general goal is to write a tool which evaluates several SAT features at the same
time and retrieve them for comparison with other problems. A SAT feature
is expected to be computable in linear time and memory with the number of variables
and number of clauses. But a suggested limit is only given with polynomial complexity
for evaluation algorithms. Sticking to this convention implies that evaluation
of satisfiability must not be necessary to evaluate a SAT feature under the
assumption that \cPneqNP. Hence the number of valid models cannot be a SAT feature
as far as satisfiability needs to be determined. But no actual hard boundary for
runtime requirements is given. Previous work has shown that expensive algorithms
can provide useful data in a small time frame if they are limited to a constant
subproblem size.

The most similar resource I found looking at SAT features was the
SATzilla project~\cite{satzilla2004,satzilla2008} in 2012. The authors systematically defined
138 SAT~features categorized in 12 groups. The features themselves are not
defined formally, but an implementation is provided bundled with example data.

\begin{table}[!h]
  \begin{center}
    \begin{tabular}{rl}
      \small{POSNEG\_RATIO\_CLAUSE\_mean} & ratio of positive to negative clauses, mean
    \end{tabular}
    \caption{SAT features defined by SATzilla}
    \label{tab:satzilla-features}
  \end{center}
\end{table}

Many SAT solvers collect feature values to improve algorithm selection,
restart strategies and estimate problem sizes. Recent trends to apply Machine
Learning to SAT solving imply feature evaluation. SAT features and the resulting
satisfiability runtime are used as training data for Machine Learning. One example
using SAT features for algorithm selection is ASlib~\cite{aslib}.

However, most of these SAT features are performance-driven.
Examples for performance-driven SAT features include the number of
restarts within a certain time frame or evaluation of local minima.



POSNEG-RATIO-CLAUSE-mean

In the following section we want to evaluate SAT features and
compare test cases.

\section{SAT features in comparison}

\begin{prop}
  The set of public benchmarks in SAT competitions between 2008 and 2015
  represent average SAT problems
\end{prop}

Define a large set of SAT features.
Present data.
Categorize data.

%\newcommand\satf{\begingroup\catcode`_=12\xxx}
%\newcommand\xxx[1]{\texttt{#1}\endgroup}
\newcommand{\satf}{\begingroup \catcode`_=12 \dosatf}
\newcommand{\dosatf}[1]{\texttt{#1}\endgroup}
\newcommand\sath[1]{\hline \multicolumn{2}{|c|}{#1} \\ \hline}
\begin{table}[!ht]
  \begin{tabular}{rl}
    \sath{Given a set of clauses, return a subset of clauses satisfying given criterion}
    \satf{clauses_allLitsNeg}     & all literals are negative \\
    \satf{clauses_oneLitNeg}      & exactly one literal is negative \\
    \satf{clauses_geqOneLitNeg}   & more than one literal is negative \\
    \satf{clauses_allLitsPos}     & all literals are positive \\
    \satf{clauses_oneLitPos}      & exactly one literal is positive \\
    \satf{clauses_geqOneLitPos}   & more than one literal is positive \\
    \satf{clauses_length1}        & clause contains exactly one literal (\enquote{unit clause}) \\
    \satf{clauses_length2}        & clause contains exactly two literals \\
    \satf{clauses_unique}         & clause did not yet occur \\
    \satf{clauses_tautological}   & clause contains some literal and its negation \\
    \sath{Given a set of literals/variables, return Boolean property}
    \satf{literals_existential}   & literal does not occur negated \\
    \satf{literals_unit}          & literal occurs in clause of length 1 \\
    \satf{literals_contradiction} & literal occurs with its negation on one clause \\
    \satf{literals_1occ}          & literal occurs only in one clause once \\
    \satf{literals_2occs}         & literal occurs two times in clauses \\
    \satf{literals_3occs}         & literal occurs three times in clauses \\
    \satf{variables_unit}         & variable occurs in clause of length 1 \\
    \sath{Given a set of clauses, return real number based on this clause}
    \satf{clauses_mapLength}      & number of literals in clause \\
    \satf{clauses_mapRatioPosNeg} & number of positive literals divided by total number of literal \\
    \satf{clauses_mapNumPos}      & number of positive literals in clause \\
    \sath{Given one clause, return Boolean property}
    \satf{clauselits_someEx}      & any is literal existential \\
    \satf{clauselits_allEx}       & all literals are existential \\
    \satf{clauselits_someUnit}    & contains unit variable \\
    \satf{clauselits_someContra}  & contains contradiction variable \\
    \satf{clauselits_all1occ}     & all variables occur only once in all clauses \\
    \satf{clauselits_all12occ}    & all variables occur only once or twice in all clauses \\
    \sath{Given all clauses, return the following property} \\
    \satf{concomp_variable}       & number of connected components where \\
                                  & two variables are in the same component \\
                                  & iff they occur in at least one clause together \\
    \satf{concomp_literal}        & number of connected components where \\
                                  & two literals are in the same component \\
                                  & iff they occur in at least one clause together \\
    \satf{xor2_count}             & Number of clause pairs ($a \lor b$, $\neg a \lor b$) \\
                                  & for two variables $a$ and $b$ \\
  \end{tabular}
\end{table}

