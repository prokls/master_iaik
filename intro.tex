\renewcommand*\chappic{img/intro.pdf}
\renewcommand*\chapquote{}
\renewcommand*\chapquotesrc{}
%
\chapter{Introduction}
\label{ch:intro}
\section{Overview}
\label{sec:intro-overview}
%
Hash functions are used as cryptographic primitives in many applications and protocols.
They take an arbitrary input message and provide a hash value. Input message and hash value
are considered as byte strings in a particular encoding.
The hash value is of fixed length and satisfies several properties which make it useful
in a variety of applications.

In this thesis, we consider the hash algorithms MD4 and SHA-256.
Our goal is to find hash collisions using differential cryptanalysis.
We define differences between two messages and determine actual bits
such that the two messages result in the same hash value. \\
This whole equation system will be modelled as a satisfiability problem.
A SAT solver reports satisfiability if and only if the particular
differences can be resolved and an actual hash collision is found.
We introduce a bit condition notation which allows us to visualize
such differential states. Verification is done by several SAT solvers
and we compare their runtime. Because the Boolean functions modelled in
a CNF have a major influence on the runtime, we investigate several
approaches and compare them.

Based on experience with these kind of problems with previous heuristic search tools
we aim to apply best practices to a satisfiability setting.
We will discuss, which SAT techniques lead to best performance characteristics
for our MD4 and SHA-256 testcases.

\section{Thesis Outline}
\label{sec:intro-outline}
%
This thesis is organized as follows:

\begin{description}
\item[In Chapter~\ref{ch:intro},] we briefly introduce basic subjects of this
  thesis. We explain our high-level goal involving hash functions and SAT solvers.

\item[In Chapter~\ref{ch:hash},] we introduce the MD4 and SHA-256 hash functions.
  Certain design decisions imply certain properties which can be used in differential
  cryptanalysis. We discuss those decisions in this chapter after a formal definition
  of the function itself. Beginning with this chapter we develop a theoretical notion
  of our tools.

\item[In Chapter~\ref{ch:dc},] we discuss approaches of differential cryptanalysis.
  We start with work done by Wang, et al. and followingly introduce differential
  notation to simplify representation of differential states. This way we can easily dump
  hash collisions.

\item[In Chapter~\ref{ch:sat},] we discuss SAT solving techniques. We discuss how
  the problem needs to be encoded and give a brief overview over used SAT solvers.
  This includes a customized lingeling version by Armin Biere for our purposes.

\item[In Chapter~\ref{ch:features},] we define SAT features which help us to
  classify SAT problems. This is a small subproject we did to look at properties
  of resulting DIMACS CNF files.

\item[In Chapter~\ref{ch:enc},] we discuss how we represent a problem (i.e. the hash
  function and a differential characteristic) as SAT~problem. This ultimatively
  allows us to solve the problem using a SAT solver.

\item[In Chapter~\ref{ch:results},] we present the result of our work.
  Runtimes are the main part of this chapter, but also results of Chapter~\ref{ch:features}
  are presented.

\item[In Chapter~\ref{ch:summary},] we conclude and discuss future work based on our results.
\end{description}
